% --------------------------------------------------------------
% This is all preamble stuff that you don't have to worry about.
% Head down to where it says "Start here"
% --------------------------------------------------------------
 
\documentclass[12pt]{article}
 
\usepackage[margin=1in]{geometry} 
\usepackage{amsmath,amsthm,amssymb,scrextend}
\usepackage{fancyhdr}
\pagestyle{fancy}

 
\newcommand{\N}{\mathbb{N}}
\newcommand{\Z}{\mathbb{Z}}
\newcommand{\I}{\mathbb{I}}
\newcommand{\R}{\mathbb{R}}
\newcommand{\Q}{\mathbb{Q}}
\renewcommand{\qed}{\hfill$\blacksquare$}
\let\newproof\proof
\renewenvironment{proof}{\begin{addmargin}[1em]{0em}\begin{newproof}}{\end{newproof}\end{addmargin}\qed}
% \newcommand{\expl}[1]{\text{\hfill[#1]}$}
 
\newenvironment{theorem}[2][Theorem]{\begin{trivlist}
\item[\hskip \labelsep {\bfseries #1}\hskip \labelsep {\bfseries #2.}]}{\end{trivlist}}
\newenvironment{lemma}[2][Lemma]{\begin{trivlist}
\item[\hskip \labelsep {\bfseries #1}\hskip \labelsep {\bfseries #2.}]}{\end{trivlist}}
\newenvironment{problem}[2][Problem]{\begin{trivlist}
\item[\hskip \labelsep {\bfseries #1}\hskip \labelsep {\bfseries #2.}]}{\end{trivlist}}
\newenvironment{exercise}[2][Exercise]{\begin{trivlist}
\item[\hskip \labelsep {\bfseries #1}\hskip \labelsep {\bfseries #2.}]}{\end{trivlist}}
\newenvironment{reflection}[2][Reflection]{\begin{trivlist}
\item[\hskip \labelsep {\bfseries #1}\hskip \labelsep {\bfseries #2.}]}{\end{trivlist}}
\newenvironment{proposition}[2][Proposition]{\begin{trivlist}
\item[\hskip \labelsep {\bfseries #1}\hskip \labelsep {\bfseries #2.}]}{\end{trivlist}}
\newenvironment{corollary}[2][Corollary]{\begin{trivlist}
\item[\hskip \labelsep {\bfseries #1}\hskip \labelsep {\bfseries #2.}]}{\end{trivlist}}
 
 \title{
Introduction to Mathematical thinking\\
\vspace{3 mm}
{\large PROOFS \\}
\vspace{3 mm}
{\Large  Deepak Kumar\\}
}
 
\begin{document}
 
% --------------------------------------------------------------
%                         Start here
% --------------------------------------------------------------

\lhead{Introduction to Mathematical thinking}

\rhead{\today}
 
\maketitle
 
\begin{problem}{1} %You can use theorem, proposition, exercise, or reflection here.  Modify x.yz to be whatever number you are proving
Say whether the following is true or false and support your answer by a proof.
$$(\exists m \in \N )(\exists n \in \N)(3m+5n = 12)$$
\end{problem}

\begin{proof}
The statement is false.\\
To show that the statement $ 3m + 5n = 12 $ is never true, we'll exhaust all posibilities of $m$ and $n$.\\ \\
Consider all $n \geq 2$, clearly for any value of $m \in \N$ we'll have $3m + 5n \geq 13$. Hence we need to consider only $n = 1$. \\ \\
Now let's assume that $(\exists m \in \N)$ that satisfies our equation. We'll do a proof by contradiction here.\\
For  $n = 1$, our equation becomes: $$3m + 5 = 12$$
$$3m = 7$$
But this means that $3|7$, which is false. Hence our assumption is false.\\
Thus, there is no such $m$ and $n$ which satisfies the original equation.
\end{proof}
 \\
 \\
 
 \begin{problem}{2} Say whether the following is true or false and support your answer by a proof: The sum of any five consecutive integers is divisible by 5 (without remainder).
\end{problem}

\begin{proof}
The statement is true.\\
To show this we'll use 5 arbitrary consecutive integers and prove that their sum is divisible by 5. \\ \\
Let $n$ be an arbitrary integer. So our 5 consecutive integers are $n, n+1, n+2, n+3, n+4$. Note that all these are arbitrary. \\ \\
Let the sum of these numbers be denoted by $S$. \\
Sum of these 5 numbers = $S$ = $(n) + (n+1) + (n+2) + (n+3) + (n+4)$
$$ = 5n + 10$$
$$ = 5(n + 2)$$
This means that $5|S$.
Hence we've proved that the statement is true.
\end{proof}
 \\
 \\
 
 
 
 
\begin{problem}{3} Say whether the following is true or false and support your answer by a proof: For any integer n, the number $ n^2 + n + 1 $ is odd.
\end{problem}

\begin{proof}
The statement is true.\\
To show this we'll use division theorem. \\ \\
By division theorem any integer n can be expressed as $2k$ or $2k+1$ where $k \in \Z$ \\ \\
Case 1: If $n = 2k$, then $$ n^2 + n + 1 =  (2k)^2 + (2k) + 1 $$
$$ = 4k^2 + 2k + 1 = 2(2k^2 + k) + 1$$. 
This number is $1$ more than a multiple of $2$. Hence, it is odd. \\ \\
Case 2: If $n = 2k+1$, then $$n^2 + n + 1 = (2k+1)^2 + (2k+1) +1$$
$$ = 4k^2 + 4k + 1 + 2k + 1 + 1$$
$$ = 4k^2 + 6k + 2 + 1 = 2(2k^2 + 3k + 1) + 1$$.
This number is $1$ more than a multiple of $2$. Hence, it is odd.\\ \\
Hence in both the cases we have $n^2 + n + 1$ as odd. Thus exhaustively the term is always odd.
\end{proof}
 \\
 \\
 
 
 
\begin{problem}{4}  Prove that every odd natural number is of one of the forms $4n + 1$ or $4n + 3$, where n is an integer.
\end{problem}

\begin{proof}
The statement is true.\\
To show this we'll use division theorem. \\ \\
By division theorem any integer k can be expressed as $4n$ or $4n+1$ or $4n+2$ or $4n+3$ where $n \in \Z$ \\ \\
Clearly $4n$, $4n+2$ are even and $4n+1$, $4n+3$  are odd. But we can express every number $k \in \Z$ in these forms.
Hence any odd number is of the form $4n+1$ or $4n+3$.
\end{proof}
 \\
 \\
 
 
 \begin{problem}{5}  Prove that for any integer n, at least one of the integers $n, n + 2, n + 4$ is divisible by 3.
\end{problem}

\begin{proof}
The statement is true.\\
To show this we'll use division theorem. \\ \\
By division theorem any integer $n$ can be expressed as $3k$ or $3k+1$ or $3k+2$ where $k \in \Z$. We'll show that if n is in any of these forms, then one of $n, n+2, n+4$ is divisible by 3 \\ \\
Case 1: $n=3k$\\
Clearly $n$ is divisible by 3.\\ \\
Case 2: $n=3k+1$ \\
So, $n+2 = 3k+2+1 = 3k+3 = 3(k+1)$.\\
Clearly $n+2$ is divisible by 3.\\ \\
Case 3: $n=3k+2$ \\
So, $n+4 = 3k+2+4 = 3(k+2)$ \\
Clearly $n+4$ is divisible by 3.\\ \\
Hence exhaustively one of $(n), (n+2), (n+4)$ is divisible by 3. Hence proved.
\end{proof}
 \\
 \\
 
 \begin{problem}{6} A classic unsolved problem in number theory asks if there are infinitely many pairs of ‘twin primes’, pairs of primes separated by 2, such as 3 and 5, 11 and 13, or 71 and 73. Prove that the only prime triple (i.e. three primes, each 2 from the next) is 3, 5, 7.
\end{problem}

\begin{proof}
We've already shown in previous problem that for any $n, n \in \Z$, one of $n, n+2, n+4$ is divisible by 3.\\
For this problem too, we have that any prime triplet will be of the form $n, n+2, n+4$. But one of these is divisible by 3 by previous proof. \\ \\
Hence for $n > 3$, one of $n, n+2, n+4$ is divisible by 3 and hence it can't be prime. Thus another prime triplet can't exist.
\end{proof}
 \\
 \\
 
 \begin{problem}{7} Prove that for any natural number n,
 $$2 + 2^2 + 2^3 + ... + 2^n = 2^{n+1} - 2$$
\end{problem}

\begin{proof}
We'll prove this using algebraic manipulation.\\ \\
Let $$S_n = 2 + 2^2 + 2^3 + ... + 2^n$$
$$2S_n = 2^2 + 2^3 + ... + 2^n + 2^{n+1}$$
Subtracting first equation from second we get: 
$$2S_n - S_n = 2^{n+1} - 2$$
Hence Proved.
\end{proof}
 \\
 \\
 
 
\begin{problem}{8} Prove (from the definition of a limit of a sequence) that if the sequence $\{a_n\}_{n=1}^{\infty}$ tends to limit $L$
as $n \to \infty$, then for any fixed number $M > 0$, the sequence $\{M a_n\}_
{n=1}^{\infty}$ tends to the limit $ML$.
\end{problem}

\begin{proof}
By definition of limit, for a given $\epsilon > 0$, we can find an $N$ such that $n \geq N \Rightarrow |a_n - L| < (\epsilon/M)$\\ \\
Thus, $n \geq N \Rightarrow |Ma_n - ML| = M.|a_n - L| < M.\epsilon/M = \epsilon$ \\
Thus $\{M a_n\}_
{n=1}^{\infty}$ tends to the limit $ML$. Hence proved
\end{proof}
 \\
 \\
 

\begin{problem}{9} Given an infinite collection $A_n, n = 1, 2, 3,...$ of intervals of the real line, their intersection is defined to be $\cap_{n=1}^{\infty} A_n = \{ x|(\forall n)(x \in A_n)\}$. Give an example of a family of intervals $A_n, n=1,2,...,$ such that $A_{n+1} \subset A_n$ for all $n$ and $\cap_{n=1}^{\infty} A_n = \phi$. Prove that your example has the stated property.
\end{problem}

\begin{proof}
Let $A_n = (0, 1/n)$. Clearly $\cap_{n=1}^{\infty} A_n \subseteq A_1 = (0,1)$. \\
Hence any element of the intersection must be a member of $(0,1)$. But if $x \in (0,1)$, we can find a natural number $n$ such that $1/n < x$. \\
Then $x \notin A_n,$ so $x \notin \cap_{n=1}^{\infty} A_n$. Thus $\cap_{n=1}^{\infty} A_n = \phi$. \\
Hence proved.
\end{proof}
 \\
 \\
 
\begin{problem}{10} Give an example of family of intervals $A_n, n=1,2,...,$ such that $A_{n+1} \subset A_n$ for all $n$ and $\cap_{n=1}^{\infty} A_n$ consists of a single real number. Prove your example has stated property.
\end{problem}

\begin{proof}
Let $A_n = [0,1/n)$. Clearly $0 \in \cap_{n=1}^{\infty} A_n$. \\ 
Using argument shown in previous question, we can show  that no other member is in the intersection. \\ \\
Hence $\cap_{n=1}^{\infty} A_n = \{0\}$.
Hence proved.
\end{proof}
 \\
 \\
 

% -------------------------------------------------------------
%     You don't have to mess with anything below this line.
% --------------------------------------------------------------
 
\end{document}
